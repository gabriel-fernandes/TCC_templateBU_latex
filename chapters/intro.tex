%% intro.tex
%%
%% Copyright 2017 Evandro Coan
%% Copyright 2012-2016 by abnTeX2 group at http://www.abntex.net.br/
%%
%% This work may be distributed and/or modified under the
%% conditions of the LaTeX Project Public License, either version 1.3
%% of this license or (at your option) any later version.
%% The latest version of this license is in
%%   http://www.latex-project.org/lppl.txt
%% and version 1.3 or later is part of all distributions of LaTeX
%% version 2005/12/01 or later.
%%
%% This work has the LPPL maintenance status `maintained'.
%% The Current Maintainer of this work is the Evandro Coan.
%%
%% The last Maintainer of this work was the abnTeX2 team, led
%% by Lauro César Araujo. Further information are available on
%% https://www.abntex.net.br/
%%
%% This work consists of a bunch of files. But originally there ware 3 files
%% which are renamed as follows:
%% Renamed the `abntex2-modelo-include-comandos` to `chapters/chapter_1.tex`
%% Renamed the `abntex2-modelo-trabalho-academico.tex` to `chapters/intro.tex`
%% Renamed the `abntex2-modelo-references.bib` to `aftertext/modelo-ufsc-references.bib`
%%
%% This file was originally the main template file, however this main file was
%% split into several new files, which are respectively drastically changed,
%% except this files which contains most of the main documentation message.
%%

% ------------------------------------------------------------------------
% ------------------------------------------------------------------------
% abnTeX2: Modelo de Trabalho Academico (tese de doutorado, dissertacao de
% mestrado e trabalhos monograficos em geral) em conformidade com
% ABNT NBR 14724:2011: Informacao e documentacao - Trabalhos academicos -
% Apresentacao
% ------------------------------------------------------------------------
% ------------------------------------------------------------------------

% The \phantomsection command is needed to create a link to a place in the document that is not a
% figure, equation, table, section, subsection, chapter, etc.
% https://tex.stackexchange.com/questions/44088/when-do-i-need-to-invoke-phantomsection
\phantomsection

% https://tex.stackexchange.com/questions/5076/is-it-possible-to-keep-my-translation-together-with-original-text
\chapter{\lang{Introduction}{Introdução}}
\phantomsection

Um dos marcos mais importantes na linha evolutiva das subestações elétricas foi a digitalização dos dados de medição e proteção oriundos dos transformadores de instrumentação, bem como a própria digitalização de todas as redes de controle e monitoração. À partir deste momento, as redes de comunicação entre os equipamentos da subestação deixaram de ser totalmente analógicas e foram substituídas gradativamente por protocolos digitais que trazem maior robustez, confiabilidade e segurança. Da mesma forma, os sinais de corrente e tensão vindos dos transformadores de instrumentação também passaram a ser discretizados e amostrados em pacotes, seguindo normas internacionais de estrutura, precisão e organização. As subestações que possuem esta características implementadas são chamadas de Subestações Elétricas Digitais.

Neste contexto, as \textit{Merging Units} são equipamentos essenciais para a interface entre os transformadores de instrumentação e os IEDs (\textit{Intelligent Electronic Devices}, como relés de proteção, gravadores digitais (\textit{Digital Recorders}) e localizadores de falhas (\textit{Fault Locators}). Estes equipamentos são responsáveis pela ``ponte'' entre a parte analógica e a parte digital de uma subestação elétrica. Na captura dos sinais de tensão e corrente advindos dos transformadores, os erros máximos de magnitude expostos na documentação do fabricante devem ser respeitados ao longo de toda a faixa de aquisição do equipamento.

%adicionar mais informações sobre merging units // tendência da digitalização das subestações // de onde vai e pra onde vem

Uma vez que estes erros máximos não são garantidos apenas pelo projeto de \textit{hardware} do equipamento, a motivação deste trabalho surge da necessidade de calibração das placas de aquisição analógica de corrente e tensão das \textit{Merging Units}. Nesta esfera, a norma IEC 61850-7-2 \cite{IEC61850_7-2} traz os Sampled Values como formato padrão para sinais elétricos e é através destes que este trabalho pretende calcular os ganhos e \textit{offsets} necessários para manter os canais de aquisição de tensão e corrente dentro das especificações propostas pelo fabricante, também em consonância com o que é proposto pela norma IEC 61869-13 \cite{IEC61869-13}. 

\section{Objetivos}

Com este plano de fundo, este trabalho tem como objetivo geral propor uma metodologia de calibração para as placas de aquisição analógica de \textit{Merging Units}, de forma a reduzir os erros de magnitude das leituras e adequar os canais aos padrões pré-estabelecidos.

Para atender ao objetivo geral, podem ser elencados os seguintes objetivos específicos:

\begin{itemize}
    \item Captura e processamento dos pacotes de \textit{Sampled Values} \cite{IEC61850-9-2} gerados pela \textit{Merging Unit} à partir de sinais de corrente e tensão aplicados nos seus canais de aquisição analógica;
    \item Conversão dos pacotes \textit{Sampled Values} em arquivos \textit{COMTRADE} \cite{comtrade1992} e avaliação dos seus erros em relação ao sinal aplicado antes das calibrações;
    \textit Implementação dos métodos de calibração;
    \item Utilização dos \textit{COMTRADEs} gerados como entrada para os métodos de calibração, de forma a obter como saída os coeficientes de linearização;
    \item Aplicação dos coeficientes nos \textit{COMTRADEs}, a fim de avaliar os resultados de cada um dos métodos.
\end{itemize}


\section{Estrutura do Trabalho}

Este trabalho é dividido em três capítulos: o primeiro introduz cada um dos métodos utilizados para a calibração dos canais de aquisição analógica da \textit{Merging Unit}, mostrando seus equacionamentos, vantagens e desvantagens. O segundo capítulo mostra como a metodologia proposta neste trabalho, utilizando os métodos do primeiro capítulo, é utilizada para obter os coeficientes de calibração para os canais de corrente e de tensão. Por sua vez, o terceiro capítulo mostra os resultados da aplicação da metodologia de calibração proposta em uma \textit{Merging Unit} real, utilizando sinais em 50 Hz e 60 Hz.









