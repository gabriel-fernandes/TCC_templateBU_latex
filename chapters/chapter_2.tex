
% The \phantomsection command is needed to create a link to a place in the document that is not a
% figure, equation, table, section, subsection, chapter, etc.
% https://tex.stackexchange.com/questions/44088/when-do-i-need-to-invoke-phantomsection
\phantomsection

% Multiple-language document - babel - selectlanguage vs begin/end{otherlanguage}
% https://tex.stackexchange.com/questions/36526/multiple-language-document-babel-selectlanguage-vs-begin-endotherlanguage
\begin{otherlanguage*}{brazil}

\chapter{Metodologia}

  %  \begin{flushright}
 %       \englishword{\showfont}
 %  \end{flushright}
\label{cap_metodologia}

\section{Ambiente de testes} %Ambiente
% Topologia de testes geral para a subestação e também a minha proposta (usada na proposta de TCC)
% Detalhes dos protocolos estilo el
%olhar posição da legenda nas imagens

Para a implementação das metodologias de testes utilizadas neste trabalho, foi necessário reproduzir em laboratório o ambiente de utilização real de uma \textit{Merging Unit}, isto é: foi necessário reproduzir o ambiente da subestação elétrica digital no qual o equipamento está inserido. O esquema exposto na figura \ref{fig:diag_substation} ilustra como é este sistema:

\begin{figure}[H]
    \centering
    \includegraphics[width=12cm]{pictures/diag_substation_tcc.png}
    \caption{Exemplo de utilização de uma \textit{Merging Unit} em uma subestação elétrica digital}
    \label{fig:diag_substation}
\end{figure}

Neste exemplo, a \textit{Merging Unit} serve como interface entre os transformadores de instrumentação e os IEDs da rede, gerando o fluxo de \textit{Sampled Values} conforme com a norma IEC-61850 \cite{IEC61850_7-2} a partir dos sinais de corrente e tensão aplicados. IED, do inglês \textit{Intelligent Electronic Device} ou Dispositivo Eletrônico Inteligente, é qualquer dispositivo que incorpora um ou mais processadores com a capacidade de receber ou enviar/controlar dados de ou para uma fonte externa de sinal (e.g., relés digitais, disjuntores) \cite{substation_basics}. Já \textit{Sampled Values} \cite{sv_typhoonhil} é o protocolo padrão utilizado para comunicação ente as \textit{Merging Units} e os IEDs. Este protocolo é do tipo \textit{publisher/subscriber}, onde o \textit{publisher} (neste caso a \textit{Merging Unit}), envia mensagens pela rede \textit{Ethernet} periodicamente com intervalos de tempo precisamente definidos. Estas mensagens contém o valor amostrado de cada um dos sinais aplicados nas placas de aquisição analógica da \textit{Merging Unit}. O intervalo de tempo depende de dois fatores: a frequência do sinal medido e a quantidade de amostras por ciclo de sinal. A norma IEC 61850-9-2LE \cite{IEC61850_7-2} define dois valores para a quantidade de amostras por ciclo: 80 para equipamentos de proteção e 256 para equipamentos de medição. Para a frequência do sinal adquirido não há restrições, sendo 50 e 60 Hz os valores mais comumente utilizados. O \textit{subscriber}, representado neste exemplo pelos \textit{IEDs}, ``subscreve'' aos pacotes de \textit{Sampled Values} buscando receber informações sobre um determinado dispositivo da rede.

Para reproduzir esse mesmo ambiente em laboratório, são substituídos os transformadores por uma fonte de tensão e corrente e o IED por um computador, que receberá e processará os \textit{Sampled Values}. A figura \ref{fig:met_acq} ilustra esse ambiente.

\begin{figure}
    \centering
    \includegraphics[width=12cm]{pictures/met_acq.png}
    \caption{Ambiente em laboratório para aquisição de dados com \textit{Merging Unit}}
    \label{fig:met_acq}
\end{figure}

Após a aquisição dos \textit{Sampled Values} pelo computador via interface de rede \textit{Ethernet}, se faz necessária a conversão deste fluxo de dados em um formato padrão para oscilografias em sistemas elétricos. Para tanto, é escolhido o formato COMTRADE \cite{comtrade1992} \cite{C37.111-2013}, que é o acrônimo para \textit{Common Format for Transient Data Exchange}, ou Formato Comum para Troca de Dados Transientes. Este formato é composto, no mínimo, por dois arquivos: um arquivo de configuração com extensão .cfg, que descreve os canais de aquisição contidos no COMTRADE e suas características, como frequência nominal do canal, quantidade de canais amostrados, unidade de medição, etc. O quadro \ref{quad:quadro_confcomtra} ilustra um exemplo de arquivo de arquivo de configuração de COMTRADE.

\begin{quadro}[htb]
\caption[Exemplo do arquivo de configuração do COMTRADE]{Exemplo do arquivo de configuração do COMTRADE.}
\label{quad:quadro_confcomtra}
\begin{tabular}{|l|}
%\hline
Test,518,1999 <CR/LF>\\
12,6A,6D <CR/LF>\\
1,P Va-g,,,kV, 0.14462,0.0000000000,0,–2048,2047,2000,1,P <CR/LF>\\
2,P Vc-g,,,kV, 0.14462,0.0000000000,0,–2048,2047,2000,1,P <CR/LF>\\
3,P Vb-g,,,KV, 0.14462,0.0000000000,0,–2048,2047,2000,1,P <CR/LF>\\
4,P Ia,,,A,11.5093049423,0.0000000000,0,–2048,2047,1200,5,P <CR/LF>\\
5,P Ib,,,A,11.5093049423,0.0000000000,0,–2048,2047,1200,5,P <CR/LF>\\
6,P Ic,,,A,11.5093049423,0.0000000000,0,–2048,2047,1200,5,P <CR/LF>\\
1,Va over,,,0 <CR/LF>\\
2,Vb over,,,0 <CR/LF>\\
3,Vc over,,,0 <CR/LF>\\
4,Ia over,,,0 <CR/LF>\\
5,Ib over,,,0 <CR/LF>\\
6,Ic over,,,0 <CR/LF>\\
60 <CR/LF>\\
1 <CR/LF>\\
6000.000,885 <CR/LF>\\
20/07/2005,17:38:26.663700 <CR/LF>\\
20/07/2005,17:38:26.687500 <CR/LF>\\
ASCII <CR/LF>\\
1\\
\hline
\end{tabular}
%\fonte{Teste. -- \showfont}
\end{quadro}

O outro arquivo imprescindível para o COMTRADE é o que contém as próprias amostras, com extensão .dat. Este arquivo pode estar codificado em ASCII com representação decimal separada por vírgulas ou em binário. Os arquivos em ASCII e em binário possuem estrutura idêntica por linha, contendo o número da amostra, o intervalo de tempo em relação ao início do registro em $\mu s$, as medidas analógicas e também as digitais. No modo binário, as informações numéricas estão em formato \textit{little-endian} de 16 bits, enquanto no formato ASCII estas são separadas por vírgula e por uma quebra de linha ao final de cada amostra. O quadro \ref{quad:quadro_dat1comtra} exemplifica essa estrutura.


\begin{quadro}[htb]
\caption[Exemplo de uma linha do arquivo de dados do COMTRADE]{Exemplo de uma linha do arquivo de dados do COMTRADE.}
\label{quad:quadro_dat1comtra}
\centering
\begin{tabular}{|c|}
%\hline
5, 667, –760, 1274, 72, 61, –140, –502,0,0,0,0,1,1 <CR/LF>\\
\hline
\end{tabular}
%\fonte{Teste. -- \showfont}
\end{quadro}

Este ambiente de testes de calibração é inspirado do processo que acontece na própria fábrica da \textit{Merging Unit}, durante a etapa de industrialização. Na fábrica, as \textit{Merging Units} são montadas, testadas e, antes de serem enviados aos clientes, passam por uma primeira etapa de calibração dos canais de aquisição analógica utilizando uma fonte de referência com suficiente precisão em relação aos erros máximos permitidos pelo fabricante. Após a obtenção dos coeficientes de calibração para cada um dos canais, o equipamento passa por um processo de \textit{Burn-In}: o mesmo é acomodado em um forno com temperatura ajustável durante 12 horas, de forma a acelerar o processo de \textit{aging} dos componentes. Após esta etapa, a \textit{Merging Unit} é calibrada uma vez mais, onde então os coeficientes desta segunda calibração sobrescrevem os coeficientes da primeira no equipamento. Neste momento, a \textit{Merging Unit} está pronta para ser enviada para o cliente.

\section{Geração de Sinais e Aquisição pela \textit{Merging Unit}}

%Para capturar os dados necessários para a aplicações dos algoritmos de calibração, se faz necessária a montagem de uma configuração de testes capaz de aplicar sinais previamente programados e com precisão adequada nos canais de tensão e corrente das placas de aquisição analógica da \textit{Merging Unit}. Estes sinais serão condicionados pelo circuito de  instrumentação de entrada da placa e, posteriormente, serão transformados em um fluxo de dados segundo o protocolo \textit{Sampled Values}, conforme a norma IEC-61850 \cite{IEC61850_7-2}, para posteriormente serem capturados por um computador. Este protocolo é usado para a troca de informações entre a \textit{Merging Unit} e os \textit{IEDs - Intelligent Electronic Devices}, ou Dispositivos Eletrônicos Inteligentes, em uma subestação elétrica digital com rede \textit{Ethernet} \cite{sv_typhoonhil}. Um \textit{IED} é qualquer dispositivo que incorpora um ou mais processadores com a capacidade de receber ou enviar/controlar dados de ou para uma fonte externa de sinal (e.g., relés digitais, disjuntores) \cite{substation_basics}.

Na configuração de testes proposta, será utilizada uma fonte de sinais de tensão e corrente programável com precisão maior que as placas de aquisição analógica da \textit{Merging Unit} em análise, de forma a reduzir os erros inerentes ao  sinal aplicado. A partir destes sinais, a \textit{Merging Unit} deverá capturar e processar em tempo real gerando como saída um fluxo de \textit{Sampled Values}. Estes fluxo de dados deverá ser direcionado a um computador para processamento posterior via interface de rede \textit{ethernet}.

\section{Recebimento dos Pacotes de \textit{Sampled Values} e reconstrução dos sinais aplicados}

Já com estes pacotes de \textit{Sampled Values} recebidos pelo computador, é momento de processar esses dados de forma a poder discernir e reconstruir os sinais para cada um dos canais de aquisição analógica da \textit{Merging Unit}. O primeiro passo deste processamento é a conversão desses pacotes em COMTRADEs. Através deste formato, cada uma das amostras presentes nos dados gerados pela \textit{Merging Unit} torna-se um valor numérico de corrente ou tensão, permitindo representar os sinais aplicados pela fonte de tensão e corrente. Na figura \ref{fig:sig_corr_reconst}, é observada essa reconstrução de um sinal de corrente de 50mA RMS:

\begin{figure}[H]
    \centering
    \includegraphics[width=14cm]{pictures/sig_corr_reconst.png}
    \caption{Exemplo de oscilografia construída a partir de COMTRADE.}
    \label{fig:sig_corr_reconst}
\end{figure}

\section{Cálculo dos erros de magnitude dos sinais reconstruídos}

Após a reconstrução das amostras para cada um dos canais de aquisição, é calculado o valor quadrático médio ou RMS (do inglês \textit{Root Mean Square}) por ciclo, como exposto na equação \ref{eq:24}:

\begin{equation}\label{eq:24}
RMS(k) = \sqrt{\frac{1}{N} \sum_{i=0}^{N - 1} |u(Nk + i)|^2}
\end{equation} 

Onde N representa o número de amostras por ciclo do sinal avaliado e k é o índice que representa o ciclo atual do sinal. A função $u(Nk + i)$ representa os sinais amostrados de corrente e tensão em análise.

Estes valores são muito informativos pois restringem o número de amostras utilizadas como entrada para os algoritmos de calibração de cada canal, sem perder informações substanciais dos sinais originais. 

Neste ponto, é possível calcular os erros de magnitude entre o valor RMS dos sinais aplicados pela fonte e cada um dos valores RMS obtidos por ciclo. De forma a aumentar a variedade dos resultados e trazer mais informações, para cada sinal aplicado são salvos o maior, o menor e a média entre os erros. 

\section{Aplicação dos Métodos de Calibração e avaliação dos Resultados}

Para obter os valores de entrada para os métodos de calibração  descritos no capítulo \ref{cap_algoritmos}, são analisados os valores RMS para cada ciclo de medição e escolhidos os valores que possuam maior erro de magnitude em relação ao valor de referência aplicado pela fonte de tensão e corrente. De forma a ampliar os resultados, também serão utilizados como valores de entrada os valores RMS que possuam o menor erro em relação ao valor de referência e também a média entre os valores RMS de cada ciclo de sinal aplicado.

Com estes valores em mãos, faz-se mister a aplicação destes nos algoritmos, obtendo assim os coeficientes $\alpha$ e $\beta$ de linearização para cada um dos métodos. Estes coeficientes são então aplicados nos valores de entrada, com o intuito de verificar o impacto da calibração dos canais de aquisição sobre os sinais de tensão e corrente amostrados.

A fim de encontrar o melhor entre os algoritmos propostos, são analisados os novos erros de magnitude entre os sinais aplicados pela fonte e os sinais calibrados, bem como figuras de mérito inerentes a este tipo de linearização, como o coeficiente de determinação $R^2$. 

Com o resultado desta análise, é então possível avaliar qual das metodologias de calibração para os canais de aquisição analógica da \textit{Merging Unit} trouxe melhores resultados, isto é: qual reduziu mais os seus erros em relação ao sinal aplicado.

De forma a sumarizar os processos descritos nesta seção, a figura \ref{fig:diag_met_aval} ilustra cada um dos passos acima descritos.

\begin{figure}[H]
    \centering
    \includegraphics[width=14cm]{pictures/diag_met_avalia.png}
    \caption{Diagrama dos procedimentos descritos na seção de aplicação dos métodos de calibração e avaliação dos resultados.}
    \label{fig:diag_met_aval}
\end{figure}

\end{otherlanguage*}

