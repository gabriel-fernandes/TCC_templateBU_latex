
% The \phantomsection command is needed to create a link to a place in the document that is not a
% figure, equation, table, section, subsection, chapter, etc.
% https://tex.stackexchange.com/questions/44088/when-do-i-need-to-invoke-phantomsection
\phantomsection

% ---
\chapter{\lang{Conclusões}{Conclusões}}
\phantomsection

   Levando em conta as considerações feitas no capítulo \ref{cap_resultados}, torna-se possível verificar a importância do emprego dos métodos de calibração para que as leituras executadas pelos canais de corrente e tensão diminuam os erros em relação ao sinal aplicado pela fonte. 
   
   Utilizando como valores de entrada dos métodos os valores de corrente RMS nos quais o erro máximo por ciclo foi encontrado, as calibrações foram capazes de reduzir os erros das medições para dentro dos limites fornecidos pelo fabricante em sinais de 60Hz. Em sinais de 50Hz, apenas o método  de mínimos quadrados ordinários se mostrou insuficiente no ponto de 50 mA, ainda assim, por apenas 0.03mA (10.1\% a mais que o permitido). Os outros métodos obtiveram resultados suficientes. Apesar dos erros médios entre os métodos serem semelhantes, é bastante perceptível a vantagem do método de desvios absolutos em relação aos outros para o sinal mais baixo de corrente aplicado. Isto evidencia a aptidão deste método em relação aos \textit{outliers}, uma vez que o ponto de 50 mA pode ser considerado um ponto com estas características. Outra particularidade interessante desta análise é que, para a corrente nominal da placa de aquisição (1 A), o algoritmo que gerou menor erro foi o método de Mínimos Quadrados Ponderados, demonstrando que, quando levado em conta na calibração os limites máximos informados pelo fabricante, a performance mostra-se superior na amplitude mais importante de toda a faixa de aquisição. Estes resultados expõem o caráter linear das medições utilizando os canais de corrente, característica esta que é essencial neste tipo de interface.
   
   Já na análise do canal de tensão, os resultados das calibrações não foram suficientes para compensar os sinais aplicados em nenhum dos testes efetuados. Isto evidencia uma pronunciada não-linearidade nas aquisições deste canal. De forma a tentar ainda mais um recurso visando diminuir os erros, foram empregadas linearizações de segunda ordem para este canal e estas trouxeram resultados promissores, principalmente para 60 Hz. Apesar da \textit{Merging Unit} utilizada não permitir calibração com coeficientes de segunda ordem, esta técnica foi utilizada de forma a tentar propor uma solução para o problema encontrado. Entretanto, mesmo neste caso, ainda não existe uma margem de segurança razoável entre o erro obtido e o erro máximo em alguns pontos. Todavia, a diminuição dos erros em relação à calibração em primeira ordem aponta uma direção no sentido de diminuir os erros deste canal, visando tornar as medições mais próximas do proposto pelo fabricante. 
  
